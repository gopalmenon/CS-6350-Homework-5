\section{Kernels}
\label{sec:q2}

\begin{enumerate}  


\item  ~[15 points] If $K_1(\bx,\bz)$ and $K_2(\bx,\bz)$ are both valid kernel
  functions. In this question, you will prove that certain functions
  of kernels are valid kernels. 

  [Hint: For both the proofs below, use the the definition of a kernel
  as a dot product in a high dimensional space.]

\begin{enumerate}
\item ~[5 points] Show that the product of two kernels is a kernel.
  That is, show that $K$ in the expression below is a kernel:
  \begin{equation*}
    K(\bx,\bz) = K_1(\bx,\bz) K_2(\bx,\bz)
  \end{equation*} 

Since $K_1$ and $K_2$ are kernels, according to Mercer's condition, for every finite set $\left \{ x_1, x_2, \ldots \right \}$, for any real valued choice of $c_1, c_2, \ldots$, we have

  \begin{equation}
   \sum_i \sum_j c_i c_j K_1 \left ( \bx_i, \bx_j \right ) \geq 0
   \label{mercer1}
  \end{equation} 

  \begin{equation}
   \sum_i \sum_j c_i c_j K_2 \left ( \bx_i, \bx_j \right ) \geq 0
   \label{mercer2}
  \end{equation} 
  
  For $K_1(\bx,\bz) K_2(\bx,\bz)$ to be a Kernel, the following needs to be true
  
  \begin{equation}
   \sum_i \sum_j c_i c_j K_1 \left ( \bx_i, \bx_j \right ) K_2 \left ( \bx_i, \bx_j \right ) \geq 0
   \label{mercer3}
  \end{equation} 
  
  Let $c_j K_1 \left ( \bx_i, \bx_j \right ) =  k_j$. Since $c_j$ can take any real values, it follows that $k_j$ can take any real value. So we can now reduce the requirement of the inequality \ref{mercer3} to
  
    \begin{equation}
   \sum_i \sum_j c_i k_j K_2 \left ( \bx_i, \bx_j \right ) \geq 0
   \label{mercer4}
  \end{equation} 
  
  However requirement \ref{mercer4} is equivalent to inequality \ref{mercer2}, which is true. This means that $K_1(\bx,\bz) K_2(\bx,\bz)$ is a valid kernel.
  
\item ~[10 points] Show that a polynomial over a kernel that is
  constructed using positive coefficients is a kernel. That is, if $P$
  is any polynomial with positive coefficients, show that $K$ below is
  a kernel:
  \begin{equation*}
    K(\bx,\bz) =P( K_1(\bx,\bz))
  \end{equation*} 
Hint: You may need show  $ K(\bx,\bz) = \alpha K_1(\bx,\bz) +  \beta K_2(\bx,\bz)$ is a valid kernel and use the conclusion in the previous question.
\end{enumerate} 
  
We have already shown in the previous answer that the product of two kernels is also a kernel. That is, if $K_1(\bx, \bz)$ is a valid kernel, then $K_1(\bx, \bz)K_1(\bx, \bz)$. From this it follows that $\left(K_1(\bx, \bz)\right)^3$ is a kernel since $\left(K_1(\bx, \bz)\right)^3 = \left( K_1(\bx, \bz) K_1(\bx, \bz)\right) K_1(\bx, \bz)= K_1'(\bx, \bz)K_1(\bx, \bz)$, where 
$K_1'(\bx, \bz) =  K_1(\bx, \bz) K_1(\bx, \bz)$. We already know that $K_1'(\bx, \bz)K_1(\bx, \bz)$ is a valid kernel since its the product of two kernels. This means that 
$\left(K_1(\bx, \bz)\right)^{n+1}$ is a kernel if $\left(K_1(\bx, \bz)\right)^n$ is a kernel. Using induction we can therefore show that $\left(K_1(\bx, \bz)\right)^n$ is a kernel for any whole number $n$.\\

A polynomial over a kernel that is constructed using positive coefficients can be shown as a combination of many terms of the form $\alpha K_1(\bx, \bz) + \beta K_2(\bx, \bz) + \ldots$. Using Mercer's condition, we need to show that 

    \begin{equation}
 \sum_i \sum_j c_i k_j \left(\alpha K_1 \left ( \bx_i, \bz_j \right ) +\beta K_2 \left ( \bx_i, \bz_j \right ) + \ldots \right) \geq 0
   \label{mercer5}
  \end{equation} 

    \begin{equation}
\alpha \sum_i \sum_j c_i k_j K_1 \left ( \bx_i, \bz_j \right ) +\beta \sum_i \sum_j c_i k_j K_2 \left ( \bx_i, \bz_j \right ) + \dots  \geq 0
   \label{mercer6}
  \end{equation} 

Since $K_1(\bx, \bz)$ and $K_2(\bx, \bz)$ and other terms are kernels, and coefficients $\alpha$, $\beta$ and others are positive, relation \ref{mercer6} is true. Which means that a polynomial over a kernel that is constructed using positive coefficients is a kernel.

\item~[10 points] Given two examples $\bx \in \Re^2$ and $\bz \in
  \Re^2$, let
  \begin{equation}
    K(\bx,\bz) = 15\left(\bx^T\bz\right)^2 \exp\left(-||\bx - \bz||^2\right)
  \end{equation}
  Prove that this is a valid kernel function.
  
According to Mercer's condition, for every finite set $\left \{ x_1, x_2, \ldots \right \}$, for any real valued choice of $c_1, c_2, \ldots$, the following needs to be true
  
  \begin{equation*}
\begin{aligned}
   \sum_i \sum_j c_i c_j 15\left(\bx_i^T\bz_j\right)^2 \exp\left(-||\bx_i - \bz_j||^2\right) &\geq 0\\
\end{aligned}
\end{equation*}

We know that $\exp\left(-||\bx_i - \bz_j||^2\right)$ will always be positive. Let $\exp\left(-||\bx_i - \bz_j||^2\right) = k_{ij}$. So the inequality that must be true is

  \begin{equation*}
\begin{aligned}
  15k_{ij} \sum_i \sum_j c_i c_j \left(\bx_i^T\bz_j\right)^2  &\geq 0
\end{aligned}
\end{equation*}
  
 If we define $\left(\bx_i^T\bz_j\right)^2$ as $(K(\bx_i, \bz_j))^2$, then we can represent kernel $(K(\bx_i, \bz_j))^2$ as another kernel $K'(\bx_i, \bz_j)$ since we have already shown that any power of a kernel is also a kernel. We now have to show that
 
   \begin{equation*}
\begin{aligned}
  15k_{ij} \sum_i \sum_j c_i c_j K'(\bx_i, \bz_j)  &\geq 0
\end{aligned}
\end{equation*}
 
 which is true by Mercer's condition since $K'(\bx_i, \bz_j)$ is a kernel.
 
 \item ~(\textbf{For 6350 students})[10 points]An valid kernel can always be  expressed as inner product. Prove that the Gaussian  kernel

$$K(\bx, \bz) = \exp(\frac{-\| x -z\|^2}{2 \sigma^2})$$
 can be written down as the inner product of an feature space with infinite dimension. Hint:  You may do some expansion and  then show the middle factor can be expanded as a power series.
 
   \begin{equation*}
\begin{aligned}
  K(\bx, \bz) &= \exp\left(\frac{-\| x -z\|^2}{2 \sigma^2}\right)\\
  &= \exp\left(\frac{-x^2 + 2xz -z^2}{2\sigma^2}\right)\\
  &= \exp\left(\frac{-x^2}{2\sigma^2}\right) \exp\left(\frac{2xz}{2\sigma^2}\right) \exp\left(\frac{-z^2}{2\sigma^2}\right)\\
  &= \exp\left(\frac{-x^2}{2\sigma^2}\right) \left(1+ \frac{\frac{xz}{\sigma^2}}{1!}  + \frac{\left(\frac{xz}{\sigma^2}\right)^2}{2!} + \frac{\left(\frac{xz}{\sigma^2}\right)^3}{3!}+ \ldots \right) \exp\left(\frac{-z^2}{2\sigma^2}\right)\\
  &= \exp\left(\frac{-x^2}{2\sigma^2}\right) \left(1+ \frac{1}{\sqrt{1!}}\frac{x}{\sigma } \frac{1}{\sqrt{1!}}\frac{z}{\sigma } +  \frac{1}{\sqrt{2!}}\left(\frac{x}{\sigma }\right)^2 \frac{1}{\sqrt{2!}}\left(\frac{z}{\sigma } \right)^2+  \ldots \right) \exp\left(\frac{-z^2}{2\sigma^2}\right)\\
\end{aligned}
\end{equation*}

This can be seen to be a product in a different dimension between

   \begin{equation*}
\begin{aligned}
\phi(x) = \exp(\frac{-x^2}{2\sigma^2}) \left(1, \frac{x}{\sqrt{1!}\sigma}, \frac{x^2}{\sqrt{2!}\sigma^2} , \frac{x^3}{\sqrt{3!}\sigma^3} , \ldots\right)
\end{aligned}
\end{equation*}
 
 and 
 
   \begin{equation*}
\begin{aligned}
\phi(z) = \exp(\frac{-z^2}{2\sigma^2}) \left(1, \frac{z}{\sqrt{1!}\sigma}, \frac{z^2}{\sqrt{2!}\sigma^2} , \frac{z^3}{\sqrt{3!}\sigma^3} , \ldots\right)
\end{aligned}
\end{equation*}
 
 Which means $K(\bx, \bz)$ can be written down as the inner product of a feature space with infinite dimensions. 
\end{enumerate}

%%% Local Variables:
%%% mode: latex
%%% TeX-master: "hw"
%%% End:
